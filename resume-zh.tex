%%
%% Copyright (c) 2018-2019 Weitian LI <wt@liwt.net>
%% CC BY 4.0 License
%%
%% Created: 2018-04-11
%%

% Chinese version
\documentclass[zh]{resume}

% File information shown at the footer of the last page
\fileinfo{%
  \faCopyright{} 2018--2019 Huifeng Chen,
  \creativecommons{by}{4.0},
  \githublink{chenhuifeng}{resume},
  \faEdit{} \today
}

\name{慧峰}{陈}

\keywords{Spark, Linux, Programming, Java, Shell, DevOps, Ansible}

% \tagline{\icon{\faBinoculars}} <position-to-look-for>}
% \tagline{<current-position>}

% \photo{<height>}{<filename>}

\profile{
  \mobile{135-0937-6427}
  \email{huifeng\_chen@foxmail.com} \\
  %\github{chenhuifeng} \\
  \university{福建师范大学}
  \degree{软件工程 \textbullet 硕士}
  \birthday{1992-08-15}
  \home{湖北 \textbullet 黄冈}
  % Custom information:
  % \icontext{<icon>}{<text>}
  % \iconlink{<icon>}{<link>}{<text>}
}

\begin{document}
\makeheader

%======================================================================
% Summary & Objectives
%======================================================================
{\onehalfspacing\hspace{2em}%
计算机专业本科,软件工程专业硕士,有扎实的计算机基础知识,有2年的工作经验,熟悉JAVA开发,擅长数据采集,有大数据开发经验,对互联网行业非常感兴趣,并希望从事相关工作。
\par}

%======================================================================
\sectionTitle{教育背景}{\faGraduationCap}
%======================================================================
\begin{educations}
    \education%
    %{2018.06}%
    {2015.09}%
    {福建师范大学}%
    {数学与计算机学院}%
    {软件工程}%
    {硕士}

  \separator{0.5ex}
  \education%
    {2015.06}%
    [2011.09]%
    {福建工程学院}%
    {计算机系}%
    {计算机科学与技术}%
    {学士}
\end{educations}


%======================================================================
\sectionTitle{专业技能}{\faCogs}
%======================================================================
\begin{itemize}
  \item 熟悉Java语法,多线程、集合等基础框架,熟悉Java编码规范并有code review相关工作经验,对JVM原理有初步的理解,包括内存模型、垃圾回收机制
  \item 熟悉Redis集群的搭建,熟悉备份策略,了解高并发下缓存穿透缓存雪崩解决方案
  \item 熟练使用Mysql关系型数据库,有过Hbase、 Hive、 Impala开发经验
  \item 熟悉Kettle数据采集,并了解Sqoop, Flume数据采集组件
  \item 熟悉Elasticsearch、Logstash、Kibana而成的日志收集系统以及搜索系统
   \item 熟悉应用服务器软件Docker等容器配置和部署,熟悉Linux系统
  \item 熟练掌握Spring Cloud、MyBatis等主流开源框架,以及Gradle、Maven等项目构建工具
\end{itemize}

%======================================================================
\sectionTitle{工作经历}{\faBriefcase}
%======================================================================
\begin{experiences}
  \experience%
    [2019.07]%
    {现在}%
    {JAVA开发工程师 @ 锐捷网络股份有限公司(上市公司)}%
    [\begin{itemize}
      \item {\bfseries 项目名称} :RG-5GNR-OMC
      \item {\bfseries 项目描述}:RG-5GNR-OMC是锐捷研发的一个管理5G基站的网管应用,主要客户群体是三大运营商。
      \item {\bfseries 涉及技术}: Spring Cloud、MyBatis、Redis、MySQL、Dubbo、RocketMQ、ELK等
      \item {\bfseries 工作内容}:\\ 
      1. 对于5G基站上报的配置数据报文,使用orika完成配置信息到java对象的映射,其中采用策略设计模式解析报文。并采用Redisson分布式锁解决高并发下数据重复的问题。\\
      2. 使用keepalived 实现服务节点的高可用并配置虚拟IP,使用nginx配置负载均衡并增强安全性改访问方式为https。\\
      3. 采用Elasticsearch , Logstash, Kibana这套经典组合,提升开发	人员线上问题的错误日志定位效率
      \item {\bfseries 担任角色}:\\
      1. 负责JAVA后端开发(配置管理模块)\\
      2. 负责nginx、keepalived部署及配置 \\ 
      3. 负责搭建ELK 分布式日志系统
      \item {\bfseries 项目收获}:
      通过该项目的开发,让我对高并发,大数据量等业务场景有了自己的认识,也知道在什么样的场景下,结合Redis以及RocketMQ等中间件的使用提高服务性能效率,以及面对高并发的场景做怎样的及时应对。
    \end{itemize}]
  
  \separator{0.5ex}

  \separator{0.5ex}
  \experience%
    [2018.07]%
    {2019.07}%
    {大数据研发工程师 @ 锐捷网络股份有限公司(上市公司)}%
    [\begin{itemize}
      \item {\bfseries 项目名称} : RG-iData
      \item {\bfseries 项目描述}:RG-iData是锐捷推出的高校大数据平台,平台数据一部分来源于高校老师,学生使用的锐捷的路由器,交换机等网络设备的上网数据,另一部分来源于高校教务系统数据。该平台主要是对学生上网产生的数据进行分析,帮助学校老师实时了解学生位置,上课考勤情况,学校舆情等。
      \item {\bfseries 涉及技术}: Kettle、CDH、Spark、Kafka、Solr 、Hive、 Hbase等
      \item {\bfseries 工作内容}:\\
      1. 使用Kettle采集各大高校教务数据,搜索日志,上网记录等内容 \\ 
      2. 使用SNMP协议采集AP信息,接着把采集数据吐到Kafka中,然后通过Spark Streaming接收数据并进行处理,按周、月导出AP接入人数及AP分布情况相关信息。\\
      3. 使用Solr和Ik分词器优化了iData搜索关键词业务,提高了对学生搜索记录和上网记录的检索准确度。\\
      4. 采用React技术,使用阿里Ant Design框架,完成舆情分析前端页面开发。\\
      5. 学习阿里Java编码规范并负责iData中代码审查,主要负责idata-api及中spark中持久化部分代码审查工作,改进iData安装包。
     \item {\bfseries 担任角色}:\\
      1. 负责平台间数据对接工作 \\
      2. 负责JAVA后端开发 \\
      3. 负责部分前端开发 
      \item {\bfseries 项目收获}:
      通过该项目的开发,初次接触到敏捷开发方法在具体项目中的应用,明白了架构设计以及开发人员开发规范在软件开发过程中的重要性。对微服务设计模式有了更深一步的认识,对联调、集成测试、文档的编写、数据库的设计等一些工作有了初步的接触,培养了自己的产品意识以及owner意识。也让自己有了不同传统企业的代码和业务思想,从需求评审、视觉评审、到自己的概要设计、详细设计、开发、联调、提测、灰度、生产上线每个环节,不断收获总结。对自己的代码有了严格的要求,leader每次review代码也能发现我的成长。\\
    \end{itemize}]
\end{experiences}

%======================================================================
\sectionTitle{获奖情况}{\faCode}
%======================================================================
\begin{itemize}
  \item 2017.11 获得硕士研究生国家奖学金
  \item 2017.09 获得“华为杯”第十四届全国研究生数学建模竞赛参与奖
  \item 2016.12 福建师范大学研究生三等学业奖学金
\end{itemize}

%======================================================================
\sectionTitle{科研成果}{\faAt}
%======================================================================
\begin{itemize}
  \item 2017.06 在世界著名期刊 Software: Practice and Experience(ccf B类)上发表一篇英文论文
    \link{https://onlinelibrary.wiley.com/doi/full/10.1002/spe.2519}{\texttt{Decomposition of UML activity diagrams}}(第一作者)
  \item 2017.07 在全国软件分析测试与演化学术会议上投中一篇论文
    \link{http://www.cnki.com.cn/Article/CJFDTotal-JSJK201802013.htm}{\texttt{基于改良程序谱的软件故障定位方法}}(第四作者)
  \item 2018.03 在《软件工程》杂志发表一篇中文论文
   \link{http://www.cnki.com.cn/Article/CJFDTotal-ZGGC201803002.htm}{\texttt{UML活动图的正确性检测}}(第一作者)
\end{itemize}

\end{document}
