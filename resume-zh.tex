%%
%% Copyright (c) 2018-2019 Weitian LI <wt@liwt.net>
%% CC BY 4.0 License
%%
%% Created: 2018-04-11
%%

% Chinese version
\documentclass[zh]{resume}

% File information shown at the footer of the last page
\fileinfo{%
  \faCopyright{} 2018--2019 Huifeng Chen,
  \creativecommons{by}{4.0},
  \githublink{chenhuifeng}{resume},
  \faEdit{} \today
}

\name{慧峰}{陈}

\keywords{Spark, Linux, Programming, Java, Shell, DevOps, Ansible}

% \tagline{\icon{\faBinoculars}} <position-to-look-for>}
% \tagline{<current-position>}

% \photo{<height>}{<filename>}

\profile{
  \mobile{135-0937-6427}
  \email{ifengyxdg@gmail.com}
  \github{chenhuifeng} \\
  \university{福建师范大学}
  \degree{软件工程 \textbullet 硕士}
  \birthday{1992-08-15}
  \home{湖北 \textbullet 黄冈}
  % Custom information:
  % \icontext{<icon>}{<text>}
  % \iconlink{<icon>}{<link>}{<text>}
}

\begin{document}
\makeheader

%======================================================================
% Summary & Objectives
%======================================================================
{\onehalfspacing\hspace{2em}%
计算机专业本科,软件工程专业硕士,有扎实的计算机基础知识,有2年多的Java开发经验,擅长数据采集,热衷于数据分析,对数据挖掘,人工智能非常感兴趣,并希望从事相关工作。
\par}

%======================================================================
\sectionTitle{技能和语言}{\faWrench}
%======================================================================
\begin{competences}
  \comptence{工具}{%
     IDEA, VScode, Kettle, Ansible, Jenkins, Docker, Git, SVN, Gradle, Maven
      }
  \comptence{编程}{%
    Java, Scala, Shell
  }
  \comptence{大数据}{%
    CDH, Spark, HDFS, Impala, Hive, Hbase, Spark, Solr, Kafka, 数据仓库
  }
  \comptence{前端}{%
    React, Echart, 阿里飞冰, Tableau
  }
  \comptence{框架}{%
    Spring Cloud, Ant Design
  }
  \comptence{\icon{\faLanguage} 语言}{
    \textbf{英语} --- 读写(优良), 听说(日常交流)
  }
\end{competences}

%======================================================================
\sectionTitle{教育背景}{\faGraduationCap}
%======================================================================
\begin{educations}
    \education%
    %{2018.06}%
    {2015.09}%
    {福建师范大学}%
    {数学与计算机学院}%
    {软件工程}%
    {硕士}

  \separator{0.5ex}
  \education%
    {2015.06}%
    [2011.09]%
    {福建工程学院}%
    {计算机系}%
    {计算机科学与技术}%
    {学士}
\end{educations}

%======================================================================
\sectionTitle{计算机技能}{\faCogs}
%======================================================================
\begin{itemize}
  \item 熟悉CDH平台组件,并使用spark,scala进行数据分析
  \item 熟悉Java编码规范并有code review相关工作经验
  \item 熟悉Spirng cloud,Ant Design框架
  \item 擅长使用Kettle进行数据采集,并了解Sqoop, Flume数据采集组件
  \item 擅长使用Ansible进行脚本编排,并通过Jenkins自动化打包、部署、测试应用
\end{itemize}

%======================================================================
\sectionTitle{获奖情况}{\faCode}
%======================================================================
\begin{itemize}
  \item 2017.11 获得硕士研究生国家奖学金
  \item 2017.09 获得“华为杯”第十四届全国研究生数学建模竞赛参与奖
  \item 2016.12 福建师范大学研究生三等学业奖学金
  % \item 2016.08 2016年全国大学生物联网设计竞赛参与奖
  \item 2015.12 福建师范大学研究生三等学业奖学金
\end{itemize}

%======================================================================
\sectionTitle{科研成果}{\faAt}
%======================================================================
\begin{itemize}
  \item 2017.06 在世界著名期刊 Software: Practice and Experience(ccf B类)上发表一篇英文论文
    \link{https://onlinelibrary.wiley.com/doi/full/10.1002/spe.2519}{\texttt{Decomposition of UML activity diagrams}}(第一作者)
  \item 2017.07 在全国软件分析测试与演化学术会议上投中一篇论文
    \link{http://www.cnki.com.cn/Article/CJFDTotal-JSJK201802013.htm}{\texttt{基于改良程序谱的软件故障定位方法}}(第四作者)
  \item 2018.03 在《软件工程》杂志发表一篇中文论文
   \link{http://www.cnki.com.cn/Article/CJFDTotal-ZGGC201803002.htm}{\texttt{UML活动图的正确性检测}}(第一作者)
\end{itemize}

%======================================================================
\sectionTitle{工作经历}{\faBriefcase}
%======================================================================
\begin{experiences}
  \experience%
    [2018.07]%
    {现在}%
    {大数据开发工程师 @ 锐捷网络股份有限公司(上市公司)}%
    [\begin{itemize}
      \item RG-iData是锐捷推出的高校大数据平台
      \item 负责iData搜索关键词业务的优化,搜索关键词业务是对学生搜索记录进行分词然后提取热点词汇及敏感词汇,由于分词的结果得到的热词都是单个字或者字符,难以满足需求,采用自然语言处理\link{https://github.com/hankcs/HanLP}{\texttt{HanLP}}算法后有较大改进。
      \item 负责iData高校数据对接,使用Kettle采集各大高校教务数据,搜索日志等内容
      \item 负责iData昆明无线城市中按周、月导出AP接入人数及AP分布情况相关信息。首先使用SNMP协议采集AP信息,接着把采集数据吐到Kafka中,然后通过Spark Streaming接收数据并进行处理。
      \item 学习阿里Java编码规范并负责iData中代码审查,主要负责idata-api及中spark中持久化部分代码审查工作
      \item  改进iData安装包,采用Ansible优化打包和升级脚步,并采用Jenkins优化打包、部署步骤
    \end{itemize}]

  \separator{0.5ex}
  \experience%
    [2017.08]%
    {2017.09}%
    {软件研发实习生 @ 锐捷网络股份有限公司(上市公司)}%
    [\begin{itemize}
      \item 学习php,并设计linux下php源码保护方案,如:zend guard软件、php-beast、apc等六种方案。根据是否付费,性能耗损情况,内存占用情况,选出两种方案,使用apache中ab压力测试对这两种方案进行性能对比。
      \item 负责锐捷网络推出的“云课堂”cmr 和web端作业空间模块部分的测试
    \end{itemize}]
\end{experiences}

\end{document}
